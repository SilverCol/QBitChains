\documentclass[a4paper]{article}
\usepackage[slovene]{babel}
\usepackage[utf8]{inputenc}
\usepackage[T1]{fontenc}
%\usepackage[margin=2cm, bottom=3cm, foot=1.5cm]{geometry}
\usepackage{float}
\usepackage{graphicx}
\usepackage{amsmath}
\usepackage{subcaption}
\usepackage{hyperref}

\newcommand{\tht}{\theta}
\newcommand{\Tht}{\Theta}
\newcommand{\dlt}{\delta}
\newcommand{\eps}{\epsilon}
\newcommand{\thalf}{\frac{3}{2}}
\newcommand{\ddx}[1]{\frac{d^2#1}{dx^2}}
\newcommand{\ddr}[2]{\frac{\partial^2#1}{\partial#2^2}}
\newcommand{\mddr}[3]{\frac{\partial^2#1}{\partial#2\partial#3}}

\newcommand{\der}[2]{\frac{d#1}{d#2}}
\newcommand{\pder}[2]{\frac{\partial#1}{\partial#2}}
\newcommand{\half}{\frac{1}{2}}
\newcommand{\forth}{\frac{1}{4}}
\newcommand{\q}{\underline{q}}
\newcommand{\p}{\underline{p}}
\newcommand{\x}{\underline{x}}
\newcommand{\liu}{\hat{\mathcal{L}}}
\newcommand{\bigO}[1]{\mathcal{O}\left( #1 \right)}

\begin{document}

    \title{\sc\large Višje računske metode\\
		\bigskip
		\bf\Large Trotter-Suzukijev razcep: Simplektični integratorji}
	\author{Mitja Vodnik, 28182041}
	\date{\today}
	\maketitle

    Rešujemo klasični problem 2D neharmonskega nihala oziroma, ekvivalentno, dveh sklopljenih 1D neharmonskih nihal. Hamiltonovo funkcijo v tem
    primeru zapišemo kot:

    \begin{equation}\label{hamilton}
        H(\q, \p) = T(\p) + V(\q) = \half p_1^2 + \half p_2^2 + \half q_1^2 + \half q_2^2 + \lambda q_1^2 q_2^2,
    \end{equation}

    kjer posplošene koordinate in momente pišemo kot $\q = (q_1, q_2)$ in $\p = (p_1, p_2)$, kinetično in potencialno energijo pa kot:

    \begin{equation}\label{energies}
        T(\p) = \half p_1^2 + \half p_2^2 \quad in \quad V(\q) = \half q_1^2 + \half q_2^2 + \lambda q_1^2 q_2^2
    \end{equation}

    Kot rešitev problema nas zanimajo trajektorije $\x(t) = (\q(t), \p(t))$ v faznem prostoru, ki so določene s Hamitonovimi enačbami gibanja:

    \begin{equation}\label{equationsOfMotion}
        \der{\x}{t} = \{\x, H(\x)\} = \liu \x
    \end{equation}

    Tu Poissonov oklepaj pišemo kot:

        \begin{equation}\label{poisson}
        \{A, B\} = \sum_{j = 1}^{N} \left( \pder{A}{q_j}\pder{B}{p_j} - \pder{A}{p_j}\pder{B}{q_j} \right),
    \end{equation}

    z $\liu = \{\bullet, H\}$ pa smo označili Liuvilleov operator. Časovni razvoj zapišemo tudi s pomočjo Liuvilleovega propagatorja:

    \begin{equation}\label{propagator}
        \x(t) = U(t)\x(0), \quad U(t) = e^{t\liu}
    \end{equation}

    Osnovna ideja simplektične integracije je razcep tega propagatoja:

    \begin{equation}\label{simplectic}
        U(t) = exp\left( \tau\{\bullet, T\} + \tau\{\bullet, V\} \right)^N, \quad \tau = \frac{t}{N},
    \end{equation}

    kjer poznamo trajektorije, ki jih generirata kinetični in potencialni propagator:

    \begin{equation}\label{baseTrajectories}
        \begin{split}
            (\q^\prime, \p^\prime) &= exp\left( \tau\{\bullet, T\}\right) (\q, \p) = \left( \q + \tau\pder{T(\p)}{\p}, \p \right) \\
            (\q^\prime, \p^\prime) &= exp\left( \tau\{\bullet, V\}\right) (\q, \p) = \left( \q, \p -\tau\pder{V(\q)}{\q} \right)
        \end{split}
    \end{equation}

	\section{Trotter-Suzukijeve formule}

    Za računanje izraza \ref{simplectic} uporabimo aproksimacijo znano kot Trotter-Suzukijeve formule. Slednje se uporablja v računanju 
    eksponentne funkcije $e^{zH}$ operatorja in temeljijo ravno  na razcepu kakršen je zgoraj omenjeni. Ideja je, da začetni operator $H$ zapišemo
    kot vsoto dveh ($H = A + B$), katerih eksponentni funkciji $e^{zA}$ in $e^{zB}$ znamo učinkovito računati (Sicer ga lahko ga razdelimo na poljubno
    število delov, a za naš primer sta dovolj dva.). Trotterjeva formula nam tedaj da oceno:

    \begin{equation}\label{trotter}
        e^{z(A + B)} = \left( e^{\frac{z}{N}A}e^{\frac{z}{N}B} \right)^N \left( 1 + \bigO{\frac{z^2}{N}} \right)
    \end{equation}

    Kot vidimo, je ta formula poljubno natančna za dovolj velik $N$ (Dodatni pogoji so sicer potrebni za neomejene operatorje, a mi imamo opravka
    le z operatorji končnega ranga.). Natančnost pa se da še povečati in sicer z dodatnim razcepom operatorja in simetrizacijo formule. Tako dobljenim
    formulam pa pravimo Suzukijeve formule in jih v splošnem zapišemo kot:

    \begin{equation}\label{suzuki}
        e^{z(A + B)} = e^{c_1zA} e^{d_1zB} ... e^{c_kzA} e^{d_kzB} + \bigO{z^{p+1}}
    \end{equation}

    Formule predstavljajo razcep na $2k$ faktorjev, katerega natančnost bo padala kot $\frac{1}{N^p}$. S $k$ torej označimo dolžino, s $p$
    pa red razcepa. Seveda, si pri konstrukciji takih formul prizadevamo doseči čim večji red s kar se da majhno dolžino. Navedimo dve Suzukijevi
    formuli z realnimi koeficienti:

    \begin{equation}\label{suzuki2}
        S_2(z) = e^{\frac{z}{2}A} e^{zB} e^{\frac{z}{2}A}, \quad e^{z(A + B)} = S_2(z) + \bigO{z^{3}}
    \end{equation}

    \begin{equation}\label{suzuki4}
        \begin{split}
            S_4(z) = S_2(x_1z)S_2(x_0z)S_2(x_1z),& \quad x_0 = -\frac{2^{1/3}}{2 - 2^{1/3}}, \quad x_1 = \frac{1}{2 - 2^{1/3}}, \\
            \quad e^{z(A + B)} &= S_4(z) + \bigO{z^{5}}
        \end{split}
    \end{equation}

    Vidimo, da sta to shemi redov $p = 2$ in $p = 4$, uspe nam torej izboljšati natančnost osnovne Trotterjeve formule \ref{trotter}. Povedati je
    treba še, da formula \ref{suzuki4} uporablja negativne koeficiente, kar je lahko problematično v primeru neomejenih operatorjev. Za formulo
    enakega reda a brez tega problema je potrebno uporabiti kompleksne koeficiente, vendar to za naš problem ne bo potrebno.

    \section{Učinkovitost formul}

    Implementiral sem štiri različne sheme za računanje časovnega razvoja nihala. Tri izmed njih pri računanju uporabijo formule iz prejšnjega
    razdelka (konkretno \ref{trotter}, \ref{suzuki2} in \ref{suzuki4}), ena pa, za primerjavo, metodo Runge-Kutta 4. reda (sistem enačb za to metodo
    se dobi z zapisom 2. Newtonovega zakona). \\
    Pri oceni učinkovitiosti metode upoštevamo dva parametra: računsko zahtevnost, ki naj bo čim manjša, in natančnost, ki naj bo čim večja.
    Graf na sliki \ref{comparison} nam pove nekaj o natančnosti metod - prikazuje kako dobro se ohranja energija v odvisnosti od izbrane velikosti
    časovnega koraka $\tau$. V tem pogledu se za najboljši izkažeta Suzukijevi formuli, kjer formula 4. reda še posebej dominira. Runge-Kutta metoda
    je še nekako primerljiva, Trotterjeva pa je za uporabo že preslaba. \\
    Kar nam ta graf namiguje, je potrjeno tudi iz stališča računske zahtevnosti: Runge-Kutta, ki sploh ni najbolj natančna, je računsko najzahtevnejša
    - porabi približno 1.5-krat več časa kot Suzukijeva formula 4. reda ($S_4$) za račun pri enakih parametrih. $S_4$ je pri enakih pogojih za podoben
    faktor zahtevnejša od $S_2$, vendar doseže boljšo natančnost pri več kot 10-krat večjih $\tau$, torej imamo vse razloge, da za nadaljne izračune
    uporabljamo metodo $S_4$.

    \begin{figure}
        \centering
        \includegraphics[width = \textwidth]{comparison/comparison}
        \caption{Maksimalno relativno odstopanje od konstantne energije na eni trajektoriji ob uporabi različnih shem: 
        Trotterjeva formula (T1) \ref{trotter},
        Suzukijeva 2. reda (S2) \ref{suzuki2},
        Suzukijeva 4. reda (S4) \ref{suzuki4} in 
        Runge-Kutta 4. reda (RK4).
        (Vse trajektorije so računane do časa okoli $t = 10000$ pri parametru $\lambda = 0$.)}
        \label{comparison}
    \end{figure}

    \section{Trajektorije}

    Kot vidimo v prejšnjem razdelku, lahko s simplektično inegracijo precej natančno računamo zelo dolge trajektorije nihala - poglejmo si nekaj
    takih trajektorij pri različnih vrednostih anharmonskega parametra $\lambda$.\\
    Na sliki \ref{order} imamo tri primere pri relativno majhnem $\lambda$. V prvem primeru je $\lambda = 0$ - prostostni stopji nihata
    razklopjeno in trajektorija je elipsa. V drugih dveh primerih, ko je $\lambda$ neničelna in sta prostostni stopnji sklopljeni, so trajektorije
    že bolj zanimive.\\
    Slika \ref{chaos} pa že prikazuje prehod v kaotičen režim - trajektorija nima več urejenega poteka. Opazimo lahko, da območje, kjer se
    nihalo giblje postaja vedno bolj zvezdasto.
    
    \begin{figure}
        \centering
        \includegraphics[width = \textwidth]{trajectories/order}
        \caption{Trajektorije razvite preko časa $t = 300$ pri vrednostih parametra:
        a) $\lambda = 0$,
        b) $\lambda = 0.5$ in
        c) $\lambda = 1$.
        (Vse trajektorije so računane z začetnimi pogoji $\x = (0, \half, 1, 0)$)}
        \label{order}
    \end{figure}
    
    \begin{figure}
        \centering
        \includegraphics[width = \textwidth]{trajectories/chaos}
        \caption{Trajektorije razvite preko časa $t = 300$ pri vrednostih parametra:
        a) $\lambda = 1.5$,
        b) $\lambda = 5$,
        c) $\lambda = 50$ in
        c) $\lambda = 1000$.
        (Vse trajektorije so računane z začetnimi pogoji $\x_0 = (0, \half, 1, 0)$)}
        \label{chaos}
    \end{figure}

    \section{Poincarejevi preseki}

    Za bolj jasno sliko o tem, katera trajektorija je kaotična in katera ne si je koristno pogledati Poincarejev presek - 2D podprostor
    v sicer (v našem primeru) 4D faznem prostoru. Jaz sem si za presek izbral $(q_1, p_1)$ ravnino. Nekaotična trajektorija v takšnem preseku
    izgleda kot množica sklenjenih krivulj, kaotična pa kot nek lik iz nepovezanih točk. \\
    Za primer si na slikah od \ref{sections1} do \ref{sections4} poglejmo, kako se preseki nekaj različnih trajektorij spreminjajo s povečevanjem
    anharmonskega parametra. Na prvi sliki \ref{sections1}a), pri $\lambda = 1$, so vse trajektorije še nekaotične - preseki večinoma
    sestojijo iz nekaj disjunktnih sklenjenih krivulj. Že na naslednji sliki je postala ena od trajektorij kaotična. Na sledečih slikah
    lahko opazujemo kako, sčasoma z dovolj velikim $\lambda$ tudi ostale trajektorije postanejo kaotične - na koncu nekaotična ostane
    le še ena, ki se sicer obdrži do nekje med $\lambda = 3.25$ in $\lambda = 3.5$.
    
    \begin{figure}
        \centering
        \includegraphics[height = .9\textheight]{sections/sections1}
        \caption{Poincarejevi preseki razvita preko časa $t = 300000$ pri vrednostih parametra:
        a) $\lambda = 1$ in
        b) $\lambda = 1.5$.
        Prikazani so preseki za 8 trajektorij z različnimi začetnimi pogoji, a z enako energijo $E = 0.625$.
        Modra trajektorija iz slike a) na sliko b) postane kaotična.}
        \label{sections1}
    \end{figure}
 
    \begin{figure}
        \centering
        \includegraphics[height = .9\textheight]{sections/sections2}
        \caption{Nadaljevanje prejšnje slike \ref{sections1}. Vrednosti parametra sta:
        c) $\lambda = 1.5$ in
        d) $\lambda = 1.75$.
        Štiri trajektorije, ki postanejo kaotične, so izvzete.}
        \label{sections2}
    \end{figure}
 
    \begin{figure}
        \centering
        \includegraphics[height = .9\textheight]{sections/sections3}
        \caption{Nadaljevanje prejšnje slike \ref{sections2}. Vrednosti parametra sta:
        c) $\lambda = 2.0$ in
        d) $\lambda = 2.5$.
        Trajektoriji, ki sta na sliki e) že kaotični sta izvzeti na sliki f).}
        \label{sections3}
    \end{figure}
 
    \begin{figure}
        \centering
        \includegraphics[height = .9\textheight]{sections/sections4}
        \caption{Nadaljevanje prejšnje slike \ref{sections3}. Vrednosti parametra sta:
        c) $\lambda = 3.0$ in
        d) $\lambda = 3.25$.
        Tu je oranžna trajektorija tik pred tem, da postane kaotična.}
        \label{sections4}
    \end{figure}

    \section{Ekviparticijski izrek}

    Ekviparticijski izrek pravi, da sta v termaliziranem stanju časovni povprečji komponent kinetične energije $\langle p_1^2 \rangle$ in
    $\langle p_2^2 \rangle$ enaki. Preverimo ta izrek za naše nihalo. Časovno povprečje komponente kinetične energije zapišemo kot:

    \begin{equation}
        \langle p_j2^2 \rangle(t) = \frac{1}{t} \int_0^t p_j^2(t^\prime)dt^\prime, \quad j = 1, 2
    \end{equation}

    Časovnemu razvoju razlike tih dveh povprečij sledimo tekom trajektorije in, če izrek velja, mora iti proti nič. Nekaj takih časovnih razvojev
    preko časa $t = 300000$ je za različne vrednosti $\lambda$ prikazanih na sliki \ref{ekvipartition}. Prvi dva grafa, ustrezata trajektorijama, ki
    nista kaotični in vidimo, da zanju izrek ne velja. Ostali trije grafi pripadajo kaotičnim krivuljam, in razlika gre proti nič. Na grafu d) sicer
    vidimo velik skok, a vendar se tudi po tem vrednost vrača nazaj. Pri zelo velikem $\lambda$, na sliki e), je to približevanje ničli videti precej
    bolj stabilno.
 
    \begin{figure}
        \centering
        \includegraphics[width = \textwidth]{ekvipartition/ekvipartition}
        \caption{Časovna povprečja razlike kvadratov komponent gibalne količine. Vrednosti parametra so:
        a) $\lambda = 0$,
        b) $\lambda = 1.0$,
        c) $\lambda = 1.25$,
        d) $\lambda = 1.75$ in
        e) $\lambda = 50.0$.
        (V vseh primerih računamo z začetnimi pogoji $\x_0 = (0, \half, 1, 0)$)}
        \label{ekvipartition}
    \end{figure}

\end{document}
