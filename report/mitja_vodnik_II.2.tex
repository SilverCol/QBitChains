\documentclass[a4paper]{article}
\usepackage[slovene]{babel}
\usepackage[utf8]{inputenc}
\usepackage[T1]{fontenc}
%\usepackage[margin=2cm, bottom=3cm, foot=1.5cm]{geometry}
\usepackage{float}
\usepackage{graphicx}
\usepackage{amsmath}
\usepackage{subcaption}
\usepackage{hyperref}

\newcommand{\tht}{\theta}
\newcommand{\Tht}{\Theta}
\newcommand{\dlt}{\delta}
\newcommand{\eps}{\epsilon}
\newcommand{\thalf}{\frac{3}{2}}
\newcommand{\ddx}[1]{\frac{d^2#1}{dx^2}}
\newcommand{\ddr}[2]{\frac{\partial^2#1}{\partial#2^2}}
\newcommand{\mddr}[3]{\frac{\partial^2#1}{\partial#2\partial#3}}

\newcommand{\der}[2]{\frac{d#1}{d#2}}
\newcommand{\pder}[2]{\frac{\partial#1}{\partial#2}}
\newcommand{\half}{\frac{1}{2}}
\newcommand{\forth}{\frac{1}{4}}
\newcommand{\q}{\underline{q}}
\newcommand{\p}{\underline{p}}
\newcommand{\x}{\underline{x}}
\newcommand{\liu}{\hat{\mathcal{L}}}
\newcommand{\bigO}[1]{\mathcal{O}\left( #1 \right)}

\begin{document}

    \title{\sc\large Višje računske metode\\
		\bigskip
		\bf\Large Trotter-Suzukijev razcep: Kubitne verige}
	\author{Mitja Vodnik, 28182041}
	\date{\today}
	\maketitle

    ...

	\section{Trotter-Suzukijeve formule}

    \begin{equation}\label{trotter}
        e^{z(A + B)} = \left( e^{\frac{z}{N}A}e^{\frac{z}{N}B} \right)^N \left( 1 + \bigO{\frac{z^2}{N}} \right)
    \end{equation}

    \iffalse
    \begin{figure}
        \centering
        \includegraphics[width = \textwidth]{comparison/comparison}
        \caption{Maksimalno relativno odstopanje od konstantne energije na eni trajektoriji ob uporabi različnih shem: 
        Trotterjeva formula (T1) \ref{trotter},
        Suzukijeva 2. reda (S2) \ref{suzuki2},
        Suzukijeva 4. reda (S4) \ref{suzuki4} in 
        Runge-Kutta 4. reda (RK4).
        (Vse trajektorije so računane do časa okoli $t = 10000$ pri parametru $\lambda = 0$.)}
        \label{comparison}
    \end{figure}
    \fi

\end{document}
