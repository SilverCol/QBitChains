\documentclass[a4paper]{article}
\usepackage[slovene]{babel}
\usepackage[utf8]{inputenc}
\usepackage[T1]{fontenc}
%\usepackage[margin=2cm, bottom=3cm, foot=1.5cm]{geometry}
\usepackage{float}
\usepackage{graphicx}
\usepackage{amsmath}
\usepackage{subcaption}
\usepackage{hyperref}

\newcommand{\tht}{\theta}
\newcommand{\Tht}{\Theta}
\newcommand{\dlt}{\delta}
\newcommand{\eps}{\epsilon}
\newcommand{\thalf}{\frac{3}{2}}
\newcommand{\ddx}[1]{\frac{d^2#1}{dx^2}}
\newcommand{\ddr}[2]{\frac{\partial^2#1}{\partial#2^2}}
\newcommand{\mddr}[3]{\frac{\partial^2#1}{\partial#2\partial#3}}

\newcommand{\der}[2]{\frac{d#1}{d#2}}
\newcommand{\pder}[2]{\frac{\partial#1}{\partial#2}}
\newcommand{\half}{\frac{1}{2}}
\newcommand{\forth}{\frac{1}{4}}
\newcommand{\q}{\underline{q}}
\newcommand{\p}{\underline{p}}
\newcommand{\x}{\underline{x}}
\newcommand{\liu}{\hat{\mathcal{L}}}
\newcommand{\bigO}[1]{\mathcal{O}\left( #1 \right)}
\newcommand{\pauli}{\mathbf{\sigma}}
\newcommand{\bra}[1]{\langle#1|}
\newcommand{\ket}[1]{|#1\rangle}
\newcommand{\id}[1]{\mathbf{1}_{2^{#1}}}

\begin{document}

    \title{\sc\large Višje računske metode\\
		\bigskip
		\bf\Large Trotter-Suzukijev razcep: Kubitne verige}
	\author{Mitja Vodnik, 28182041}
	\date{\today}
	\maketitle

    Obravnavamo Heisenbergovo spinsko verigo $N$ spinov $S = \half$, kjer je $N$ sodo naravno število.
    Uporabljamo izotropen antiferomagneten ($J = -1$) model, z interakcijo le med najbližjimi sosedi:

    \begin{equation}\label{eq1}
        H = \sum_{j=1}^{N} \pauli_j \cdot \pauli_{j+1}
    \end{equation}

    Tu je $\pauli_j$ vektor Paulijevih matrik, ki delujejo na $j$-tem mestu.
    Zaradi preprostosti vpeljemo še periodičen robni pogoj $\pauli_N = \pauli_0$.\\

    Radi bi računali osnovno dinamiko in termodinamiko spinske verige, za kar je potrebno poznati delovanje
    propagatorja, tj. eksponentne funkcije Hamiltonovega operatorja $e^{zH}$ - uporabili bomo Trotter-Suzukijev razcep.
    Operator najprej razcepimo na dva dela:

    \begin{equation}\label{eq2}
            H = A + B, \quad
            A = \sum_{j=1}^{N/2} \pauli_{2j-1} \cdot \pauli_{2j}, \quad
            B = \sum_{j=1}^{N/2} \pauli_{2j} \cdot \pauli_{2j+1}
    \end{equation}

    Posamezni členi v operatorjih $A$ in $B$ komutirajo, zato lahko pišemo:

    \begin{equation}\label{eq3}
    e^{zA} = \prod_{j=1}^{N/2} exp(z\pauli_{2j-1} \cdot \pauli_{2j}), \quad
    e^{zB} = \prod_{j=1}^{N/2} exp(z\pauli_{2j} \cdot \pauli_{2j+1})
    \end{equation}

    Poznati moramo torej le eksponentne funkcije skalarnega produkta Paulijevih matrik za dva sosednja spina:

    \begin{equation}\label{eq4}
    \begin{split}
    &exp(z\pauli_{j} \cdot \pauli_{j+1}) = \id{j-1} \otimes U_{j, j+1}^{(2)} \otimes \id{N-j-1}\\
        &U_{j, j+1}^{(2)} = e^{-z}
        \begin{pmatrix}
            e^{2z} & 0 & 0 & 0 \\
            0 & ch(2z) & sh(2z) & 0 \\
            0 & sh(2z) & ch(2z) & 0 \\
            0 & 0 & 0 & e^{2z}
        \end{pmatrix}
    \end{split}
    \end{equation}

    Zgornja matrika je zapisana v lastni bazi operatorja $S_z$, $\{ |00\rangle, |01\rangle, |10\rangle, |11\rangle \}$,
    kjer $0$ označuje stanje s pozitivno, $1$ pa z negativno $z$ projekcijo spina.

	\section{Implementacija propagatorja}

    Stanja zapišimo v lastni bazi operatorja $\sigma_j^z$:

    \begin{equation}\label{eq5}
        \ket{\psi} = \sum_{s_1 \ldots s_N} \psi_{s_1 \ldots s_N} \ket{s_1, \ldots, s_N}
    \end{equation}

    Delovanje operatorja~\ref{eq4} sedaj obravnavamo po koeficientih:

    \begin{equation}\label{eq6}
        \begin{split}
            \ket{\psi^\prime} &= exp(z\pauli_{j} \cdot \pauli_{j+1}) \ket{\psi}\\
            \psi_{s_1^\prime \ldots s_N^\prime}^\prime &=
            \sum_{s_j, s_{s+1}} U_{(s_j^\prime s_{j+1}^\prime) (s_j s_{j+1})}
            \psi_{s_1^\prime \ldots s_{j-1}^\prime s_j s_{j+1} s_{j+2}^\prime \ldots s_N^\prime}
        \end{split}
    \end{equation}

    Indeksirajmo bazo Hilbertovega prostora z $n = 0, \ldots, 2^N - 1$, kjer binarni zapis indeksa $n$ ustreza zapisu
    pripadajočega lastnega stanja $|s_1, \ldots, s_N\rangle$:

    \begin{equation}\label{eq7}
        n = (s_N\ldots s_1)_2, \quad s_i \in \{ 0, 1 \}
    \end{equation}

    Koeficiente v razvoju~\ref{eq5} imamo naprimer shranjene v tabeli pod indeksi, ki popolnoma opisujejo konfiguracije
    pripadajočih stanj - $s_j^\prime$ in $s_{j+1}^\prime$ dobimo enostavno kot vrednosti bitov na $j$-tem in $j+1$-vem
    mestu.
    Upoštevajmo še, da ima matrika $U_{j, j+1}^{(2)}$ le različne neničelne elemente:

    \begin{equation}\label{eq8}
    \begin{split}
        U_{(00)(00)} &= U_{(11)(11)} = e^{z}\\
        U_{(01)(01)} &= U_{(10)(10)} = e^{-z}ch(2z)\\
        U_{(10)(01)} &= U_{(01)(10)} = e^{-z}sh(2z)
    \end{split}
    \end{equation}

    Iz tega je lahko videti, da v izrazu~\ref{eq6} ločimo le dva primera:

    \begin{equation}\label{eq9}
    \begin{split}
        \psi_{s_1^\prime \ldots s_N^\prime}^\prime &= e^{z} \psi_{s_1^\prime \ldots s_N^\prime},
        \quad ko \quad s_j^\prime = s_{j+1}^\prime\\
        \psi_{s_1^\prime \ldots s_N^\prime}^\prime &= e^{-z}ch(2z) \psi_{s_1^\prime \ldots s_N^\prime} + e^{-z}sh(2z)
        \psi_{s_1^\prime \ldots s_{j-1}^\prime s_{j+1}^\prime s_j^\prime s_{j+2}^\prime \ldots s_N^\prime},
        \quad sicer\\
    \end{split}
    \end{equation}

    Indeks zadnjega koeficienta v zgornjem izrazu dobimo, če indeksu $n$ obrnemo bita na mestih $j$ in $j+1$.
    Opisali smo implementacijo delovanja poljubnega faktorja v izrazu~\ref{eq3}, ki ga že znamo uporabiti za
    konstrukcijo Trotter-Suzukijevih formul.

    \section{Termalna povprečja}

    Pri računaju termalnih povprečij, nas zanima particijska funkcija:

    \begin{equation}\label{eq10}
        z(\beta) = tr \left( e^{-\beta H} \right)
        = \frac{1}{N_\psi} \sum_{l=1}^{N_\psi} \bra{\psi_l} e^{-\beta H} \ket{\psi_l}
    \end{equation}

    V zgornjem izrazu smo že zapisali Monte Carlo približek za računanje sledi: namesto po kompletnem setu lastnih
    stanj, seštevamo po setu naključnih stanj $\{ \ket{\psi_l} \}_{l=1}^{N_{\psi}}$, kar nam da poljubno natančen
    rezultat, pri dovolj velikem številu spinov $N$.\\

    Algoritem za računanje particijske funkcije je naslednji:

    \begin{enumerate}
        \item Generiramo naključno stanje $\ket{\psi_l}$: Imaginarne in realne dele koeficientov generiramo po
        normalni porazdelitvi, in nato stanje normiramo.
        \item Stanje propagiramo v imaginarnem času:
        $\ket{\psi_l^{\frac{\beta}{2}}} = e^{-\frac{\beta}{2}} \ket{\psi_l}$
        \item Prejšnji točki izvedemo za vseh $N_\psi$ stanj, in nato particijsko funkcijo izračunamo kot povprečje
        kvadratov norm dobljenih stanj:
        \begin{equation}\label{eq11}
        z(\beta) = \frac{1}{N_\psi} \sum_{l=1}^{N_\psi} \bra{\psi_l^{\frac{\beta}{2}}} \ket{\psi_l^{\frac{\beta}{2}}}
        \end{equation}
    \end{enumerate}

    Iz tako dobljene particijske funkcije lahko izračunamo prosto energijo verige v odvisnosti od inverzne temperature
    $\beta$:

    \begin{equation}\label{eq12}
        F = -\frac{1}{\beta}log(z(\beta))
    \end{equation}

    Če v točki $3$ našega algoritma izračunamo še standardni odklon $\sigma_z$, lahko ocenimo napako, ki smo jo naredili
    ko nismo seštevali po kompletni bazi:

    \begin{equation}\label{eq13}
    \sigma_F = -\frac{\sigma_z(\beta)}{\beta z(\beta)}
    \end{equation}

    Graf~\ref{slika1} prikazuje tako dobljeno prosto energijo za več različnih dolžin verige.
    Prikazane so tudi napake izračunane po zgornjem izrazu~\ref{eq13} - vidimo da so tem manjše, čim večja je dolžina
    verige.

    \begin{figure}
        \centering
        \includegraphics[width = \textwidth]{slika1.pdf}
        \caption{Graf proste energije v odvisnosti od inverzne temperature.}
        \label{slika1}
    \end{figure}

    Dalje lahko izračunamo pričakovano vrednost energije v odvisnosti od inverzne temperature:

    \begin{equation}\label{eq14}
        \langle H \rangle_\beta = -\pder{}{\beta} log(z(\beta))
    \end{equation}

    Rezultat tega računa je prikazan na sliki~\ref{slika2}.
    Vidimo, da se pričakovana vrednost energije za različno dolge verige približuje različnim konstantam, ki so kar
    energije osnovnih stanj.\\

    Če rezultate delimo z dolžino verige, dobimo pričakovano vrednost energije na delec.
    Za dovolj velike $N$ ta vrednost konvergira k energiji osnovnega stanja (na delec) za neskončno verigo - to nam
    nakazuje slika~\ref{slika3}.

    \begin{figure}
        \centering
        \includegraphics[width = \textwidth]{slika2.pdf}
        \caption{Pričakovana vrednost energije v odvisnosti od inverzne temperaure.}
        \label{slika2}
    \end{figure}

    \begin{figure}
        \centering
        \includegraphics[width = \textwidth]{slika3.pdf}
        \caption{Pričakovana vrednost energije na delec v nizkotemperaturni limiti ($\beta = 4$) v odvisnosti od
        dolžine verige.}
        \label{slika3}
    \end{figure}

    \section{Korelacijske funkcije}

    \ldots

\end{document}
